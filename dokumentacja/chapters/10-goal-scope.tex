\clearpage
\chapter{Cel i zakres pracy}
\section{Cel pracy}
\customnormalsize{
Celem pracy jest usprawnienie procesu konsultacji poprzez zwiększenie dostępności oraz ułatwienie procesu planowania i komunikacji. Niezbędne do tego jest stworzenie rozwiązania do planowwania, tworzenia terminów konsultacji oraz możliwości zapisu na te konsultacje. Wszystko to celem zwwiększenia dostępności konsultacji oraz zachęcenia studentów do korzystania z takich możliwości.
}
\customlarge{\section{Zakres pracy}}
\customnormalsize{
Zakres pracy zakłada:
\begin{enumerate}
    \item Implementację interfejsu użytkownika (UI)
    \item Utworzenie serwerowej części aplikacji przy użyciu Node.js
    \begin{itemize}
        \item Utworzenie interfejsów API do komunikacji pomiędzy serwerem a interfejsem użytkownika oraz bazą danych
        \item Wdrożenie operacji związanych z uprawnieniami użytkowników oraz autentykacją
        \item Implementacja mechanizmów uwierzytelniania i autoryzacji przy użyciu JWT (JSON Web Token)
        \item Zabezpieczenie aplikacji przed atakami typu Cross-Site Scripting (XSS)
    \end{itemize}
    \item Zaprojektowanie oraz stworzenie bazy danych przechowywującej informację o studentach, prowadzących oraz terminach konsultacji
    \begin{itemize}
        \item Tworzenie zapytań korzystając z ORM celem zmniejszenie obciążenia serwera bazodanowego
        \item Zabazpieczenie aplikacji przed atakami typu SQL Injection
    \end{itemize}
    \item Optymalizację operacji
    \begin{itemize}
        \item Wykorzystanie mechanizmów cache'owania danych celem zmniejszenia czasu ładowania stron
    \end{itemize}
    
\end{enumerate}
}