\chapter{Wstęp}
\section{Rola konsultacji w życiu akademickim}
Często tematyka rozwijana w trakcie zajęć akademickich jest skomplikowana na tyle, że nie sposób dokładnie wyjaśnić i odpowiedzieć na wszystkie pytania studentów w czasie określonym przez harmonogram. Czasem studenci chcą skorzystać z rozległej wiedzy prowadzącego, żeby ukierunkował ich w stronę rozwiązania ich problemu. Istnieją również sytuacje, w których niezbędna jest poprawa kolokwium, celem uzyskania oceny pozytywnej z przedmiotu. Właśnie dlatego istnieją konsultację. Prowadzący wyznaczają dodatkowy czas, w którym są "do dyspozycji" studentów i z chęcią, po wcześniejszym ustaleniu, im pomogą. Problem niestety często bywa z organizacją tych zajęć - ciężko dokładnie ustalić termin z prowadzącym, nie ma możliwości informowania o potrzebie przełożenia konsultacji oraz wiele podobnych przykładów. W~związku z tym istnieje potrzeba rozwiązania, które zniweluje te problemy, lub chociaż zmniejszy prawdopodobieństwo ich wystąpienia.
\section{Charakterystyka problematyki związanej z organizacją konsultacji}
Proces organizowania konsultacji może być skomplikowany ze względu na mnogość możliwych do wystąpienia problemów i wyzwań. Każdy z nich wymaga indywidualnego podejścia i unikalnego rozwiązania. Oto niektóre z kwestii związanych z tym tematem:
\begin{enumerate}
    \item Dostępność
        \begin{itemize}
            \item[] Konsultacje studenckie muszą być dostępne w godzinach i terminach dostępnych dla wszystkich studentów, zwłaszcza tych uczestniczących w kursach lub posiadających zobowiązania pozanaukowe
        \end{itemize}
    \item Logistyka
        \begin{itemize}
            \item[] Rozwiązanie zapewniające organizację konsultacji musi brać pod uwagę wyzwania związane z dostępnością prowadzących jak i liczbą dostępnych na uczelni sal
        \end{itemize}
    \item Zróżnicowane potrzeby studentów
        \begin{itemize}
            \item[] Studenci mogą posiadać różne powody, w związku z którymi chcą przyjść na konsultację. Każdy z tych powodów musi zostać potraktowany indywidualnie, ze względu np. na czas niezbędny na przeprowadzenie tej konkretnej konsultacji 
        \end{itemize}
    \newpage
    \item Komunikacja i elastyczność
        \begin{itemize}
            \item[] Charakter zdarzeń losowych sprawia, że nie jesteśmy w stanie przewidzieć ich wystąpienia. W tym przypadku, komunikacja na linii student-prowadzący często bywa utrudniona lub nieugruntowana
        \end{itemize}
\end{enumerate}
\section{Wyzwania w procesie komunikacji i planowania}
Aspekty takie jak komunikacja i planowanie konsultacji są niesamowicie istotne, gdyż to od nich zależy przebieg i efektywność konsultacji. Reiterując opisane poprzednio wyzwania, należy skupić się na analizie przyczyny ich występowania i określić sposób, w jaki należy im sprostać:
\begin{enumerate}
    \item Komunikacja
        \begin{itemize}
            \item[] Proces komunikacyjny pomiędzy prowadzącym konsultację a wyrażającymi chęć udziału w niej studentami musi być przejrzysty i łatwo dostępny. Ze względu na mnogość możliwych kanałów komunikacyjnych istnieje możliwość przeoczenia takiej wiadomości, co może skutkować, że konsultacje nie zostaną przeprowadzone. Istnieje zatem potrzeba unifikacji systemu do zapisu na konsultację z kanałem komunikacyjnym, w którym student i prowadzący mogą się porozumieć, dzięki czemu unikniemy powyżej opisanego problemu.
        \end{itemize}
    \item Planowanie
        \begin{itemize}
            \item[] Jeśli chodzi o proces planowania, należy skupić się na kilku niezbędnie ważnych aspektach, których poprawne zaadresowanie może pozwolić na usprawnienie procesu planowania:
            \begin{itemize}
                \item Kanał komunikacyjny
                    \begin{itemize}
                        \item[] Zmuszanie zarówno prowadzących jak i studentów do konkretnego kanału komunikacyjnego może okazać się złym pomysłem, na przykład ze względu na starszy wiek prowadzących i istniejącą w naszym społeczeństwie lukę cyforową. Należy jednak skupić się na tym, by ten kanał komunikacyjny został zaakceptowany przez prowadzącego jak i studentów, przez co stanie się głównym środkiem wymiany informacji w tej konkretnej sytuacji.
                    \end{itemize}
    \newpage
                \item Elastyczność w zakresie terminu konsultacji
                    \begin{itemize}
                        \item[] Ze względu na opisany wcześniej charakter zdarzeń losowych, istnieje potrzeba rozwiązania, w którym możliwa jest zmiana godziny lub odwołanie/przełożenie w czasie konsultacji. Dzięki takiemu rozwiązaniu unikniemy sytuacji, w której to jedna ze stron traci czas na przybycie na miejsce konsultacji, która się nie odbędzie 
                    \end{itemize}
                \item Odzwierciedlanie stanu faktycznego
                    \begin{itemize}
                        \item[] Systemy informatyczne powstały ze względu na potrzebę digitalizacji i odzwierciedlania stanu rzeczywistości, który w łatwy sposób można aktualizować, przez co efektywnie zastępuje kartkę papieru. Zgodnie z tą zasadą, system informatyczny musi automatycznie adaptować się do stanu w rzeczywistości, w którym na przykład, choroba prowadzącego sprawia, że wszystkie konsultacje w tym terminie się nie odbędą, a w związku z tym sale będą wolne
                    \end{itemize}
            \end{itemize}
        \end{itemize}
\end{enumerate}
\newpage
\section{Technologie zastosowane w projekcie}
Po przeanalizowaniu niezbędnych do implementacji funkcji oraz zasady działania systemu, niewątpliwe jest to, że istnieje potrzeba zaimplementowania tego rozwiązania w formie aplikacji WEB. W związku z powyższym, w tej realizacji zostaną wykorzystane następujące technologie:
\begin{itemize}
    \item HTML, CSS, Javascript
        \begin{itemize}
            \item[] Jest to absolutny standard, jeśli chodzi o współczesny rozwój aplikacji WEB. HTML [HyperText Markup Language] pozwoli na zbudowanie szkieletu strony i pozwoli na wyświetlenie treści w przeglądarce internetowej. CSS [Cascading Style Sheets] odpowiada za sferę estetyczną aplikacji, co przekłada się na większą dostępność treści dla użytkownika, np.wskazując możliwe do wykonania akcje za pomocą kolorów i kształtów. JavaScript umożliwi interakcję pomiędzy serwerem, a użytkownikiem poprzez np. możliwość reagowania na akcje podejmowanie przez użytkownika. Warto zauważyć, że zarówno React.js jak i Node.js zostały zaprojektowane w oparciu o Javascript.
        \end{itemize}
    \item React.js
        \begin{itemize}
            \item[] Bliblioteka języka Javascript, stwworzona przez programistę Facebooka, pozwalająca na tworzenie interfejsów użytkownika. React bazuje na dzieleniu elementów strony na komponenty, które potem mogą zostać używane wielokrotnie. React opiera się na JSX, czyli rozszerzeniu składni Javascriptu, co pozwala na łatwiejsze implementowanie logiki w obszarze komponentu. W momencie pisania tej pracy, wiele serwisów, takich jak np. Netflix, korzystają właśnie z Reacta, by budować własny interfejs.
        \end{itemize}
    \item Node.js
        \begin{itemize}
            \item[] Powstałe w oparciu o Javascript środowisko uruchomieniowe, które może służyć jako serwer. Przewagą użycia Node.js jest jego obsługa asynchroniczności, co jest niezmiernie ważnym aspektem w przypadku budowania aplikacji czasu rzeczywistego.
        \end{itemize}
    \item MySQL
        \begin{itemize}
            \item[] Otwartoźródłowy system do zarządzania relacyjnymi bazami danych. Zapewnia przechowywanie informacji w sposób usystematyzowany, przez co dostęp do nich jest szybki zautomatyzowany. MySQL jest rozwijane przez firmę Oracle.
        \end{itemize}
\end{itemize}
